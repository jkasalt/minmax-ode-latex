\documentclass[main.tex]{subfiles}

\begin{document}
\section{The process}
In light of \cite{luResolutionODEFramework2021} we present here a method
to obtain differential equations from discrete algorithms.
We assume that the velocity depends on the current position in the following way 
\[ \dot Z(t) = f_0(Z(t)) + sf_1(Z(t)) + s^2f_2(Z(t)) + \dots \]
To alleviate the notation we will write $Z$ instead of $Z(t)$.
We obtain a Taylor expansion
\begin{align*}
	Z(t + s) & = Z + s(f_0(Z) + sf_1(Z) + \dots)                                          \\
	     & + \frac{s^2}{2}(f'_0(Z) + sf'_1(Z) + \dots)(f_0(Z) + sf_1(Z) + \dots) + \dots
\end{align*}\label{eq:taylor-x}

On the other hand, given a numerical algorithm $z_{n+1} = \varphi(z_n, s)$
we can also express 
\section{Example: Mirror descent $O(s^r)$-resolution ODE}

On the other hand, mirror descent uses as update rule
\[ g(x, s) = \nabla\psi^{-1}(\nabla\psi(x) + s \nabla F(x)) \]
The derivatives are
\begin{align*}
	\partial_s g(x, s)    & = \nabla_s \nabla \psi^{-1}(\nabla\psi(x) + s \nabla F(x))\nabla F(x)                      \\
	\partial_{ss} g(x, s) & = \nabla^2_s \nabla \psi^{-1}(\nabla\psi(x) + s \nabla F(x)) \nabla F(x) \cdot \nabla F(x) \\
\end{align*}

We consider the Taylor expansion of $g(x,s)$ w.r.t.\ $s$ and equate the coefficients of both polynomials to find
\begin{align*}
	f_0                             & = \partial_s g(x,s)                                                                                         \\
	f_1 + \frac{1}{2}f'_0 \cdot f_0 & = \partial_{ss} g(x,s)                                                                                      \\
	\implies f_1                    & = \nabla^2_s \nabla \psi^{-1}(\nabla\psi(x) + s \nabla F(x)) \nabla F(x) \cdot \nabla F(x)                  \\
	                                & + \frac{1}{2}\nabla_s \nabla^2 \psi^{-1}(\nabla \psi (x) + s \nabla F(x))(\nabla^2 \psi(x) + s\nabla^2F(x)) \\
	                                & \ \cdot  \nabla_s \nabla \psi^{-1}(\nabla\psi(x) + s \nabla F(x))\nabla F(x)
\end{align*}
\end{document}

